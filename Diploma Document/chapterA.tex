\chapter{Gathering information and creating a plan}\label{ch:A}
\section{Plan of project}
At the planning stage of the project, we could not decide which of the methodologies we would use. Our lead PM preferred the waterfall methodology, as the project had a structured flow, with each element following the other. He also suggested supplementing the waterfall with Kanaban to make it easier to follow the project and, in case of any shifts, thanks to a flexible methodology, problems can be responded to and fixed much faster.
\\
Below you can see the Gantt chart that was made by our PM at the beginning of our project \ref{fig:plan1}. The plan and diagram did not change throughout the project. But in the course of the project, thanks to the Kanaban board, additional tasks or additions to the already completed tasks were introduced. Do not forget that every two weeks we had a meeting with {\mycoach}, so the project and its tasks were edited every two weeks.
\\
Also below you can see our Kanban table \ref{fig:knbrd}, in which we followed the process of each member of our team. All the tasks described in it have been approved by our entire team. Accordingly, each of the participants took part in its writing. As mentioned earlier, this table was created so that we can keep track of any deviations in our project or changes that our practice teacher {\mycoach} advised us to make. In particular, you can see the plan that was written on the page of our directory, which describes the detailed work of each of the members of our group \ref{fig:weeksplan}.

\begin{figure}[ht]
    \centering
    \includegraphics[scale=0.5]{plan1.png}
    \caption{Gantt chart}
    \label{fig:plan1}
\end{figure}

\begin{figure}[ht]
    \centering
    \includegraphics[scale=0.5]{plan2.png}
    \caption{Gantt chart}
    \label{fig:plan2}
\end{figure}

\begin{figure}[ht]
    \centering
    \includegraphics[scale=0.5]{plan3.png}
    \caption{Gantt chart}
    \label{fig:plan3}
\end{figure}

\begin{figure}[!htb]
    \centering
    \includegraphics[scale=0.5]{Kanbanbrd.png}
    \caption{Kanban board}
    \label{fig:knbrd}
\end{figure}
%Картинка съезжает, придумать, что с этим делать%
\begin{figure}[t]
    \centering
    \includegraphics[scale=0.5]{weeksplan.png}
    \caption{Plan on Git}
    \label{fig:weeksplan}
\end{figure}

\newpage
\section{User Personas and Stories}

Let us cite some resources form the  bibliography. This is a a good book \cite{dirac}. Let's cite Einstein's paper \cite{einstein} and Knuth's website \cite{knuthwebsite}. Knuth has also a book on algorithms \cite{knuth-fa}.

\section{Sitemap, Page descriptions}